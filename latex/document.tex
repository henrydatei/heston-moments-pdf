\documentclass[a4paper, titlepage, twoside, openright, appendixprefix, numbers=noendperiod]{scrreport}

% % % % % % % % % % % % % % % % %
% WICHTIG!
% Das Dokument ist zweiseitig formatiert, es MUSS beidseitig gedruckt werden.
% Neue Kapitel beginnen immer rechts (Option openright)
% wird "numbers=noendperiod" gelöscht, reagiert das KOMASkript Dudenkonform: bei Überschriftennummerierung keine Endpunkte ohne Anhang; mit Endpunkten, wenn es einen Anhang gibt 
% % % % % % % % % % % % % % % % %

% Laden der Zusatzpakete (aus dem Ordner "input") 
% Paket für Input Encoding
% wenn utf8 nicht funktioniert, bitte ansinew (Windows) oder applemac (Mac) benutzen.
\usepackage[utf8]{inputenc}

% Font Encoding, u.A. für die korrekte Ausgabe in PDF-Dokumenten
\usepackage[T1]{fontenc}

% Anpassung der Sprache, in diesem Fall: Deutsch in neuer Rechtschreibung
\usepackage[english]{babel}

% Einstellung der Geometry des Layouts
\usepackage[textwidth=145mm,textheight=235mm,left=35mm,top=25mm,headsep=5mm]{geometry}

% optimierte Schrift für PDF-Dokumente
\usepackage{lmodern}

% Zum einfachen Einfügen von Grafiken
\usepackage{graphicx}

% Für lang Tabellen
\usepackage{longtable}

% Erweiternde Pakete für den Formelsatz
\usepackage{amsmath, amssymb, amsthm, amsfonts}

% Erstellt Verweise in PDF-Dokumenten. Die Verweise haben die Farbe schwarz, sind also nicht extra gekennzeichnet.
\usepackage[colorlinks,linkcolor=black,citecolor=black]{hyperref} 

% Paket für Aufzählungen. Zu verwenden wie itemize.
\usepackage{enumerate}

% Anführungszeichen
\usepackage[style=german]{csquotes} 

% "schöne" Tabellen
\usepackage{booktabs}

% Deckblatt für die Seminararbeit. Metadaten in titlepage.tex anpassen.
\usepackage{VOSTitle}

% Paket für die Selbstständigkeitserklärung, nutzt die Metadaten in titlepage.tex
\usepackage{VOSStatement}

% Für Zeilenabstände
\usepackage{setspace}

% Blindtext
\usepackage{lipsum}

% Paket für das Abkürzungsverzeichnis
\usepackage{acronym}

% Paket für die Zusammenfassung nach dem Titelblatt
\usepackage[style]{abstract}

% Paket für angepasste Bibliografie-Stile
\usepackage{bibgerm}

\usepackage{fancyhdr}
% Optional: Gleitobjekte nicht in andere Abschnitte fließen lassen 
%(Doku:http://mirror.informatik.uni-mannheim.de/pub/mirrors/tex-archive/macros/latex/contrib/placeins/placeins-doc.pdf) 
%\usepackage{placeins}

\usepackage[style=apa]{biblatex}
\usepackage{parskip}

% Laden weiterer Einstellungen
% einige andere Einstellungen oder Ergänzungen

% Festlegen des Titel-Stils der abstract-Umgebung
\renewcommand{\abstitlestyle}[1]{{\Large\bfseries\sffamily\noindent #1}\hfill}

% Änderung der Nummerierung der Formeln	
\numberwithin{equation}{section} 

% anderthalbfacher Zeilenabstand 
\onehalfspacing

% Globaler Seitenstil

\pagestyle{headings}
\fancyhf{} %
\pagestyle{fancy} %
\fancyfoot[LE,RO]{\thepage} %
\fancyhead[LO]{\nouppercase\rightmark} %
\fancyhead[RE]{\nouppercase\leftmark}%




% Parameter für das Titelblatt und die Selbstständigkeitserklärung
% % % % % %
% Bitte beachten Sie die Hinweise zum Ausfüllen.
% % % % % % 

% Lehrstuhl, an dem die Arbeit geschrieben wurde
\professur{Chair of Econometrics and Statistics}

% Art der wissenschaftlichen Arbeit
\thesistype{Diploma Thesis}

\fak{\enquote{Friedrich List} Faculty of Transport and Traffic Sciences}

% Namen und Matrikelnummern möglicher Autoren.

% % % % % % % % % % % % % % % % % % % % % % % % % %
%   Bitte die Autoren DER REIHE NACH auffüllen	  %
% % % % % % % % % % % % % % % % % % % % % % % % % %

% Bei nur einem Autor muss authorOne ausgefüllt werden
\authorOne{Henry Haustein}
\matrikelAuthorOne{4685025}

% Hat die Arbeit zwei Autoren, muss authorTwo ausgefüllt werden
\authorTwo{} 
\matrikelAuthorTwo{}

% Bei drei Gruppenmitgliedern ist auch authorThree zu belegen
\authorThree{} 
\matrikelAuthorThree{}

% NUR, FALLS TATSÄCHLICH BENÖTIGT, ANSONSTEN LEER LASSEN
% Für das vierte Gruppenmitglied
\authorFour{}
\matrikelAuthorFour{}

% Titel der Aufgabenstellung
\title{WIP}

% Betreuer am Lehrstuhls
\betreuer{Haozhe Jiang}

% Datum der Abgabe. \today ist der heutige Tag, bitte ggfs. ändern auf den 8. Januar oder wann immer Sie abgeben
\date{XX.XX.XXXX}



\addbibresource{library.bib}

% % % % % % % % % % % % % % % %
% Beginn der document-Umgebung
% % % % % % % % % % % % % % % %
\begin{document}

% Erstellen des Titelblatts
\maketitle

\cleardoublepage

% Umstellung der Seitennummerierung auf römische Ziffern
\pagenumbering{roman}

% Zusammenfassung, wenn nicht benötigt, auskommentieren!
% \selectlanguage{ngerman}

% \begin{abstract}
% \noindent
% \lipsum[20-21]
% \end{abstract}

% FALLS BENÖTIGT: Englische Zusammenfassung (in der Regel nicht in Seminararbeiten gefordert!)
%\selectlanguage{english}
%\begin{abstract}
%\noindent
%\lipsum[20-23]
%\end{abstract}
%\selectlanguage{ngerman}

% \cleardoublepage

% Erstellen des Inhalts-, Abbildungs- und Tabellenverzeichnisses
\tableofcontents
\cleardoublepage
  
\phantomsection\addcontentsline{toc}{chapter}{List of Figures} 
\listoffigures
\cleardoublepage

\phantomsection\addcontentsline{toc}{chapter}{List of Tables}
\listoftables
\cleardoublepage

% Einfügen des Abkürzungsverzeichnisses
% Bitte die Datei listofabbreviations.tex im Order um die eigenen Abkürzungen ergänzen.
% \phantomsection\addcontentsline{toc}{chapter}{Abkürzungsverzeichnis}
% \chapter*{Abkürzungsverzeichnis}
% Der String XXXXXXXX hat keine wirkliche Bedeutung; die acronym-Umgebung verlangt einen Parameter, der angibt, wie breit die Abkürzungen in der Übersicht sein dürfen, in diesem Fall 8*X
\begin{acronym}[XXXXXXXX]
% \acro{BIP}{Bruttoinlandsprodukt}
% \acro{x}[\ensuremath{\bar{x}}]{Mittelwert}
\end{acronym}
% \cleardoublepage

% Umstellung der Seitennummerierung auf wieder auf arabische Ziffern 
\pagenumbering{arabic}

% Hier beginnen die Kapitel und der eigentliche Inhalt.
% Es empfiehlt sich, den Inhalt einzelner Kapitel in separaten Dateien zu halten und diese hier mit \input{file} einzubinden.

%Einleitungskapitel, hier sind derzeit noch nachfolgende Kapitel drin, diese besser in eigene Dateien trennen
% \section{Introduction}
\label{sec:introduction}

- In vielen Einführungsvorlesungen in Finance wird die Annahme getroffen, dass Renditen am (Aktien-)Markt normalverteilt sind. Diese Annahme ist jedoch nicht korrekt, da empirische Untersuchungen zeigen (z.B. Mandelbrot 1997), dass Renditen oft nicht normalverteilt sind. So sind Renditen oft leptokurtisch, d.h. die Verteilung hat dickere Enden als die Normalverteilung. Zudem sind Renditen oft schief, d.h. die Verteilung ist nicht symmetrisch. Diese Eigenschaften sind wichtig, da sie die Risikobewertung und das Risikomanagement beeinflussen.
- Die Idee der normalverteilten Renditen kommt von Bachelier (1900), der die These aufstellte, dass der Preis eines Assests sich wie eine Brownian Motion verhält. Diese These wurde von Black, Scholes und Merton im Jahr 1973 weiterentwickelt zum Black-Scholes-Merton-Modell, welches die Finanzwelt revolutionierte (Heimer & Arend 2008). Der britische Mathematiker Ian Steward is sogar der Meinung, dass dieses Modell verantwortlich für die Finanzkrise 2007-2008 war (Steward 2012).
- Die Schwächen des Black-Scholes-Modells sind die Annahmen über unter anderem eine konstante Volatilität und eine normalverteilte Rendite. Diese Annahmen sind empirisch nicht haltbar, da die Volatilität oft nicht konstant ist und die Renditen oft nicht normalverteilt sind. Daher wurden Modelle entwickelt, die diese Schwächen adressieren. Eines dieser Modelle ist das Heston-Modell, welches von Heston (1993) vorgestellt wurde. Das Heston-Modell ist ein stochastisches Volatilitätsmodell, d.h. die Volatilität ist nicht konstant, sondern folgt auch einem Zufallsprozess. Das Heston-Modell ist ein beliebtes Modell in der Finanzmathematik, da es die Schwächen des Black-Scholes-Modells adressiert und die empirischen Eigenschaften von Renditen besser abbildet.
- Das Heston-Modell hat keine geschlossene Lösung mehr, im Gegensatz zum Black-Scholes-Modell, daher müssen numerische Verfahren zur Lösung des Modells verwendet werden. Diese Verfahren sind aufwendig, entweder simuliert man viele Pfade oder man nutzt die charakteristische Funktion in Kombination mit einer inversen Fourier-Transformation, um eine Preisverteilung zu berechnen. Diese Preisverteilung kann dann genutzt werden, um Optionen zu bewerten (Gatheral 2011).
- Das setzt voraus, dass man bereits die Parameter des Modells kennt. Man kann das Modell an den Markt anpassen, indem man es als Least-Squares-Problem definiert, wobei die Zielfunktion ist, die Preise am Markt zu reproduzieren. Normalerweise nimmt man die Preise von vanilla options (Floc'h 2018) oder von variance swaps (Guillaume & Schoutens 2013).
- Da Verteilung der Renditen doch recht ähnlich zur Normalverteilung ist, ist es das Ziel der Arbeit zu untersuchen, inwiefern sich Expansionsverfahren für die Normalverteilung, wie z.B. die Gram-Charlier Expansion (Gram 1883, Charlier 1914) genutzt werden kann die Normalverteilung zu verändern, um die Renditen des simulierten Heston-Modells abzubilden. Dazu wird das Heston-Modell mittels Andersens (2008) QE-Verfahren für einen großen Parameterraum simuliert, Momente und Kumulanten berechnet und dann die Expansionsverfahren angewendet. Die Ergebnisse werden dann mit der theoretischen Dichte verglichen, um zu sehen, wie gut die Expansionsverfahren die Dichte approximieren können.

- Die Arbeit gliedert sich in die folgenden Bereiche: Im Kapitel \ref{sec:heston_model} wird das Heston-Modell vorgestellt, das Kapitel \ref{sec:moments} werden Grundlagen zu Momenten und Kumulanten gelegt und auf die Berechnung dieser bei Hochfrequenz-Handelsdaten eingegangen. Im Kapitel \ref{sec:expansion_methods} werden die Expansionsverfahren vorgestellt und in Kapitel \ref{sec:methodical_approach} die praktische Umsetzung der Arbeit erläutert. Es folgt Kapitel \ref{sec:results} mit Ergebnissen und die Arbeit schließt mit Kapitel \ref{sec:conclusion} ab, wo die Arbeit zusammengefasst und ein Ausblick gegeben wird.

%Anhang
% \appendix
% \phantomsection\addcontentsline{toc}{chapter}{Appendices}
% \input{input/appendices}

% Festlegen des Bibliografie-Stils, in diesem Fall die deutsche Variante des Standard-Stils plain
% Für Bachelor- und Masterarbeiten sollte biblatex oder natbib verwendet werden. Eine kurze Suche bei Google führt auf die Paketdokumentationen,
% aus der sich alles Weitere ergibt.
% \bibliographystyle{gerplain}

% Einbinden der Bibliothek
\printbibliography
\thispagestyle{empty}

% Befehl zum Erstellen der Selbstständigkeitserklärung. Achtung: Es werden die Namen der Autoren, die in titlepage.tex festgelegt wurden, herangezogen und das Datum in \date{} verwendet.
% Befehl steht nur zur Verfügung, wenn das Paket VOSStatement eingebunden ist.
\cleardoublepage

\makestatement

% % % % % % % % % % % % % % % %
% Ende der document-Umgebung
% % % % % % % % % % % % % % % %
\end{document}