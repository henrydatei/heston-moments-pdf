\chapter{Conclusion and Future Work}
\label{sec:conclusion_future_work}

- Die numerischen Fehler während der Simulation des Heston-Models mittels QE-Schema konnten identifiziert und ausgeschlossen werden. Es stellte sich heraus, dass, wenn die Feller condition zu stark nicht erfüllt war, diese Fehler auftauchten. Andersen (2008) schreibt aber in seinem Paper, dass das QE-Schema auch in solchen Fällen funktioniert, es kann sich also nur um einen Fehler in der Implementierung handeln. Tatsächlich konnte dieser Fehler zum Ende dieser Arbeit gefunden und behoben werden, die Zeit reichte aber nicht mehr aus, um diese Simulationen neu durchrechnen zu lassen. Problem war die Berechnung der Gleichung \eqref{eq:qe_dirac} und das Ziehen von $U_v$ aus der uniform distribution. Falls $U_v$ dort exakt den Wert 1 annimmt, so wird zur Berechnung von $\Psi^{-1}$ durch 0 geteilt, was zu dem Fehler führt. Das kommt aber nur zum Tragen, wenn $v_t$ sehr klein ist und bleibt, also die Feller condition stark nicht erfüllt ist. Das Ziehen von $U_v$ wurde so angepasst, dass es zwischen $10^{-10}$ und $1-10^{-10}$ liegt. Einige kurze Tests haben gezeigt, dass dieser Fehler damit behoben ist.