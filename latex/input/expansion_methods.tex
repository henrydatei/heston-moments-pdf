\section{Expansion Methods}
- expansion methods are series that approximate a probability density function
- usually they are not true densities, since they can go below zero for certain parameters
- for some parameters, they are densities, we'll explore this in the following sections and how to get from a parameter set which gives not a density to a parameter set which gives a density

\subsection{Gram-Charlier Expansion}
- entdeckt von Gram 1883 und Charlier 1914
- zwei Arten von Serien: Gram-Charlier A und Gram-Charlier B:
\begin{align}
    f_{GC,A} &\approx f(x) + \sum_{k=3}^n a_k f^{(k)}(x) \notag \\
    f_{GC,B} &\approx \psi(x)\sum_{m=0}^n b_mg_m(x) \notag
\end{align}
- auch wenn die Expansion für jede Dichte $f$ und $\psi$ geht, so ist für den Typ A $f$ die Dichte der Standardnormalverteilung
\begin{align}
    f(x) = \frac{1}{\sqrt{2\pi}}\exp\left(-\frac{x^2}{2}\right) \notag
\end{align}
und für den Typ B $\psi$ die Wahrscheinlichkeitsfunktion der Poisson-Verteilung (Mitropol'skii 2020)
\begin{align}
    \psi(x) = \frac{\lambda^x}{x!}\exp(-\lambda) \notag
\end{align}
- $f^{(k)}$ ist die $k$-te Ableitung der Dichte $f$ und es existieren Polynome $H_k$, die folgende Gleichung erfüllen:
\begin{align}
    f^{(k)}(x) = (-1)^k f(x)H_k(x) \notag
\end{align}
- Die Polynome $H_k$ sind als Hermite-Polynome (Laplace 1811, Laplace 1812, Chebychef 1860, Hermite 1864) bekannt und haben folgende Eigenschaften (Abramowitz & Stegun 1968, p. 771ff):
\begin{align}
    H_{n+1} &= x\cdot H_n(x) - H'_n(x) \notag \\
    H'_n(x) &= n\cdot H_{n-1}(x) \notag \\
\end{align}
Damit lassen sich die Hermite-Polynome rekursiv berechnen und die ersten Polynome sind:
\begin{align}
    H_{n+1}(x) &= x\cdot H_n(x) - nH_{n-1}(x) \notag \\
    H_0(x) &= 1 \notag \\
    H_1(x) &= x \notag \\
    H_2(x) &= x^2 - 1 \notag \\
    H_3(x) &= x^3-3x \notag \\
    H_4(x) &= x^4-6x^2+3 \notag \\
    H_5(x) &= x^5-10x^3+15x \notag \\
    H_6(x) &= x^6-15x^4+45x^2-15 \notag
\end{align}
- Koeffizienten $a_k$ können als Momente $r_k$ der Dichte $f$ definiert werden und so erhält man die ersten Terme der Gram-Charlier A Expansion:
\begin{align}
    f(x)_{GC,A} \approx \frac{1}{\sqrt{2\pi}\sigma}\exp\left(-\frac{(x-\mu)^2}{2\sigma^2}\right) \left[1 + \frac{\kappa_3}{6\sigma^3}H_3\left(\frac{x-\mu}{\sigma}\right) + \frac{\kappa_4}{24\sigma^4}H_4\left(\frac{x-\mu}{\sigma}\right)\right] \notag
\end{align}
dabei sind $\mu$, $\sigma^2$, $\kappa_3$ und $\kappa_4$ die ersten 4 Cumulants der zu approximierenden Verteilung. Aufgrund von \eqref{eq:cumulants_1} und \eqref{eq:cumulants_2} entsprechend $\mu$ und $\sigma^2$ den ersten beiden Cumulants $\kappa_1$ und $\kappa_2$.
- Die ersten Terme der Gram-Charlier B Expansion sind:
\begin{align}
    f(x)_{GC,B} \approx \frac{\lambda^x}{x!}\exp(-\lambda) /\left(1 + \frac{\mu_2 - \lambda}{\lambda^2}\left[\frac{x^{[2]}}{2} - \lambda x^{[1]} + \frac{\lambda^2}{2}\right] + \frac{\mu_3 - 3\mu_2 + 2\lambda}{\lambda^3}\left[\frac{x^{[3]}}{6} - \frac{\lambda}{2}x^{[2]} + \frac{\lambda^2}{2}x^{[1]} - \frac{\lambda^3}{6}\right]\right) \notag
\end{align}
wobei $\mu_i$ die zentralen Momente der zu approximierenden Verteilung sind und $x^{[i]} = x(x-1)\dots (x-i+1)$ (Mitropol'skii 2020).
- Die Gram-Charlier Expansion ist keine asympotische Expansion, weil es nicht möglich ist, den Fehler der Approximation zu ermitteln. Die Edgeworth Expansion ist allerdings eine asympotische Expansion (Cramer 1999, Section 17.6) und wird daher bevorzugt. Eine asymptotische Expansion ist eine Serie von Funktionen $f_n$, die nach einer endlichen Anzahl von Termen eine Approximation einer Funktion in einem bestimmten Punkt $\xi$ (oftmals infinte) darstellt, wenn das Argument $x$ gegen $\xi$ läuft:
\begin{align}
    f_{n+1}(x) = \mathcal{o}(f_n(x)) \quad x\to\xi \notag
\end{align}

\subsection{Edgeworth Expansion}
- beschrieben von Edgeworth 1907, er schlägt unter anderem vor, die Approximation bis zum 6. Term zu berechnen um das oben angesprochene Problem der asympotischen Expansion zu umgehen. Eine formelle Darstellung der ersten Terme findet sich z.B. in Brenn & Anfinsen (2017):
\begin{align}
    f(x)_{EW} \approx \frac{1}{\sqrt{2\pi}\sigma}\exp\left(-\frac{(x-\mu)^2}{2\sigma^2}\right) \left[1 + \frac{\kappa_3}{6\sigma^3}H_3\left(\frac{x-\mu}{\sigma}\right) + \frac{\kappa_4}{24\sigma^4}H_4\left(\frac{x-\mu}{\sigma}\right) + \frac{\kappa_5}{120\sigma^5}H_5\left(\frac{x-\mu}{\sigma}\right) + \frac{\kappa_6 + 10\kappa_3^2}{720\sigma^6}H_6\left(\frac{x-\mu}{\sigma}\right)\right] \notag
\end{align}
In dieser Arbeit nur Betrachtung der ersten 4 Cumulants, daher reduziert sich der Ausdruck auf:
\begin{align}
    f(x)_{EW} \approx \frac{1}{\sqrt{2\pi}\sigma}\exp\left(-\frac{(x-\mu)^2}{2\sigma^2}\right) \left[1 + \frac{\kappa_3}{6\sigma^3}H_3\left(\frac{x-\mu}{\sigma}\right) + \frac{\kappa_4}{24\sigma^4}H_4\left(\frac{x-\mu}{\sigma}\right) + \frac{\kappa_3^2}{72\sigma^6}H_6\left(\frac{x-\mu}{\sigma}\right)\right] \notag
\end{align}

\subsection{Cornish-Fisher Expansion}

\subsection{Saddlepoint Approximation}