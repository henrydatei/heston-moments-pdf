\chapter{The Heston Model}
\label{sec:heston_model}

\section{Model Description}

The Heston model, introduced by Heston (\citeyear{hestonClosedFormSolutionOptions1993}), is a stochastic volatility model in which volatility is not constant, as in the Black-Scholes-Merton model, but instead follows a random process. The dynamics of the model are given by the following system of stochastic differential equations:
\begin{align}
    \label{eq:heston_model_price}
    \mathrm{d}S_t &= \mu S_t\mathrm{d}t + \sqrt{v_t}S_t\mathrm{d}W_t^S \\
    \label{eq:heston_model_log_price}
    \mathrm{d}X_t = \mathrm{d}\log(S_t) &= \left(\mu-\frac{1}{2}v_t\right)\mathrm{d}t + \sqrt{v_t}\mathrm{d}W_t^S \\
    \label{eq:heston_model_variance}
    \mathrm{d}v_t &= \kappa(\theta-v_t)\mathrm{d}t + \sigma\sqrt{v_t}\mathrm{d}W_t^v \\
    \label{eq:heston_model_correlation}
    \mathbb{E}(\mathrm{d}W_t^S\mathrm{d}W_t^v) &= \rho\mathrm{d}t
\end{align}
where $\kappa$, $\theta$, and $\sigma$ are strictly positive parameters. The terms $\mathrm{d}W_t^S$ and $\mathrm{d}W_t^v$ represent the increments of Brownian motions with correlation $\rho$. The variable $S_t$ denotes the price of an asset, such as a stock, bond, or foreign exchange rate. The process $X_t$ represents the logarithm of the price process $S_t$, while $v_t$ denotes the instantaneous variance process. The parameter $\mu$ represents the drift of the price process.

The variance process follows a Cox-Ingersoll-Ross (CIR) process (\cite{coxTheoryTermStructure1985}) with mean reversion $\kappa$, long-run variance $\theta$, and volatility $\sigma$. The conditional transition probability of $v_t$ given $v_0$ is proportional to a noncentral chi-squared distributed random variable:
\begin{align}
    \label{eq:heston_model_variance_transition}
    v_t\mid v_0 &\sim c\cdot \chi^{2'}_{\nu}(\Lambda) \\
    c &= \sigma^2\left(1 - \exp(-\kappa t)\right)(4\kappa)^{-1} \notag \\
    \nu &= \frac{4\kappa\theta}{\sigma^2} \notag \\
    \Lambda &= \frac{v_0}{c}\exp(-\kappa t) \notag
\end{align}
where $\chi^{2'}$ denotes a noncentral chi-squared distribution with $\nu$ degrees of freedom and noncentrality parameter $\Lambda$ (\cite{okhrinSimulatingCoxIngersoll2022}).

\section{Characteristic Function and Density of the Heston Model}

If a random variable $X$ has a density function $f(x)$, its characteristic function $\phi(t)$ is given by
\begin{align}
    \phi(t) = \mathbb{E}(\exp(\mathrm{i}tX)) = \int_{-\infty}^{\infty} e^{\mathrm{i}tx}f(x)\mathrm{d}x \notag
\end{align}
The characteristic function always exists, even if the probability density function does not. Once the characteristic function is known, the density function can be recovered via the inverse Fourier transform:
\begin{align}
    f(x) = \frac{1}{2\pi}\int_{-\infty}^{\infty} e^{-\mathrm{i}tx}\phi(t)\mathrm{d}t \notag
\end{align}
Gatheral (\citeyear{gatheralVolatilitySurfacePractitioner2011}) derives the characteristic function of the Heston model as
\begin{align}
    \phi(t) = \exp(A + B + C) \notag
\end{align}
where
\begin{align}
    A &= \mu\cdot\tau\cdot t\cdot\mathrm{i} \notag \\
    d &= \sqrt{(\rho\sigma\mathrm{i}t - \kappa)^2 - \sigma^2(-\mathrm{i}t - t^2)} \notag \\
    g &= \frac{\kappa - \rho\sigma\mathrm{i}t - d}{\kappa - \rho\sigma\mathrm{i}t + d} \notag \\
    B &= \frac{\theta\kappa}{\sigma^2}\left(\tau(\kappa - \rho\sigma\mathrm{i}t - d) - 2\log\left[\frac{1-g\exp(-d\tau)}{1-g}\right]\right) \notag \\
    \gamma &= \frac{2\kappa\theta}{\sigma^2} \notag \\
    \label{eq:heston_model_characteristic_function_C}
    C &= \log\left(\left[\frac{2\kappa}{\sigma^2}\right]^{\gamma}\cdot \left\lbrace \frac{2\kappa}{\sigma^2} - \frac{\kappa - \rho\sigma\mathrm{i}t - d}{\sigma^2}\cdot\frac{1 - \exp(-d\tau)}{1-g\exp(-d\tau)} \right\rbrace^{-\gamma}\right)
\end{align}
where $\tau = T-t$ represents the time horizon.

A straightforward inverse Fourier transform is not suitable due to numerical instabilities at the boundaries (see Figure \ref{fig:ifft_comparison}). The simple method centers $\phi(t)$ and normalizes $f(x)$ using the step size $\Delta t$ to ensure correct amplitudes. An alternative approach employs a boundary correction, which improves the reconstruction of the density. Additionally, results are smoothed using cubic spline interpolation.

Equation \eqref{eq:heston_model_characteristic_function_C} has the drawback that it can lead to overflow errors, as the term $\left(\frac{2\kappa}{\sigma^2}\right)^\gamma$ may become excessively large, making the logarithm intractable. To mitigate this issue, the equation is reformulated as
\begin{align}
    \label{eq:heston_model_characteristic_function_C_2}
    C_{unc} = \gamma \log\left(\frac{2\kappa}{\sigma^2}\right) - \gamma \log\left(\frac{2\kappa}{\sigma^2} - \frac{\kappa - \rho\sigma t \mathrm{i} - d}{\sigma^2} \frac{1 - \exp(-d \tau)}{1 - g \exp(-d \tau)}\right)
\end{align}

\begin{figure}[h]
    \centering
    \includegraphics[width=0.8\textwidth]{img/different_ifft_methods.png}
    \caption{Comparison of different methods for the inverse Fourier transformation of the characteristic function of the Heston model ($\mu=0$, $\kappa=3$, $\theta=0.19$, $\sigma=0.4$, $\rho=-0.7$, $\tau=\frac{1}{12}$). Grid points: $N=2^{15}$}
    \label{fig:ifft_comparison}
\end{figure}

\section{Simulating the Heston Model}

The Heston model is a continuous-time model, and for simulation purposes, time must be discretized. Small timesteps require significant computational power, while large timesteps may lead to zero or even negative volatility if the so-called Feller condition is not satisfied (\cite{albrecherLittleHestonTrap2007}). The Feller condition holds if $2\kappa\theta > \sigma^2$, but Eraker et al. (\citeyear{erakerImpactJumpsVolatility2003}) show that for S\&P 500 index options, this condition is violated. Many other studies confirm this result for various markets and asset classes (e.g., \cite{changOptionPricingDouble2021,huPricingVolatilityJump2022}).

The most intuitive approach to time discretization is the Euler–Maruyama scheme:
\begin{align}
    X_{t+\Delta} &= X_t - \frac{1}{2}v_t\Delta + \sqrt{v_t}\sqrt{\Delta}Z_{X,t} \notag \\
    v_{t+\Delta} &= v_t + \kappa(\theta - v_t)\Delta + \sigma\sqrt{v_t}\sqrt{\Delta}Z_{v,t} \notag
\end{align}
where $Z \sim \mathcal{N}(0,1)$, $\Delta = T/n$, with $T$ representing the total time and $n$ the number of time steps. The correlation between $Z_{X,t}$ and $Z_{v,t}$ can be simulated as follows (\cite{andersenEfficientSimulationHeston2007}): %TODO: andere Quelle finden, nicht immer nur Andersen oder Ostap
\begin{align}
    Z_{v,t} &= \Phi^{-1}(U_1) \notag \\
    Z_{X,t} &= \rho Z_{v,t} + \sqrt{1-\rho^2}\Phi^{-1}(U_2) \notag
\end{align}
where $U_1$ and $U_2$ are independent random variables uniformly distributed on $[0,1]$, and $\Phi^{-1}$ is the inverse cumulative distribution function of the standard normal distribution.

This discretization highlights the issue of potential negative volatility (\cite{okhrinSimulatingCoxIngersoll2022}):
\begin{align}
    \mathbb{P}(v_{t+\Delta}<0 \mid v_t>0) &= \mathbb{P}\left(Z_{v,t} < \frac{-v_t-\kappa(\theta-v_t)\Delta}{\sigma\sqrt{v_t}\sqrt{\Delta}}\right) \notag \\
    &= \Phi\left(Z_{v,t} < \frac{-v_t-\kappa(\theta-v_t)\Delta}{\sigma\sqrt{v_t}\sqrt{\Delta}}\right) \notag \\
    &> 0 \notag
\end{align}
where $\Phi$ denotes the cumulative distribution function of the standard normal distribution. Various methods exist to address this issue, such as the absorption method (replacing $v_t$ with $v_t^+ = \max(0, v_t)$) or the reflection method (replacing $v_t$ with $\vert v_t\vert$). However, these modifications alter the underlying process, causing the moments of volatility to deviate from their theoretical values (\cite{okhrinSimulatingCoxIngersoll2022,tsoskounoglouSimulatingHestonModel2024}).

In the study by Okhrin et al. (\citeyear{okhrinSimulatingCoxIngersoll2022}), Andersen's Quadratic Exponential (QE) scheme is found to perform best in terms of both speed and accuracy. Andersen (\citeyear{andersenEfficientSimulationHeston2007}) approximates the noncentral chi-square distribution in \eqref{eq:heston_model_variance_transition} using a mixture of distributions: a Dirac distribution and a noncentral Gaussian distribution. For sufficiently large values of $v_t$,
\begin{align}
    \label{eq:qe_normal}
    v_{t+\Delta} = a(b+Z_v)^2
\end{align}
where $Z_v \sim \mathcal{N}(0,1)$. For smaller values of $v_t$,
\begin{align}
    \label{eq:qe_dirac}
    v_{t+\Delta} &= \Psi^{-1}(U_v, p, \beta) \\
    \Psi^{-1}(u,p,\beta) &= \begin{cases}
        0 & 0\le u\le p \\
        \beta^{-1}\ln\left(\frac{1-p}{1-u}\right) & p<u\le 1
    \end{cases} \notag
\end{align}
The parameters $a$, $b$, $p$, and $\beta$ are estimated using moment matching. The transition between the two approximation schemes is determined by a threshold $\psi_c$: if the ratio $\psi = s^2 / m^2$ (where $m$ and $s^2$ are the mean and variance of $v_{t+\Delta}$, respectively) exceeds $\psi_c$, then \eqref{eq:qe_dirac} is used; otherwise, \eqref{eq:qe_normal} is applied.

For the price process, Andersen (\citeyear{andersenEfficientSimulationHeston2007}) proposes the following discretization scheme:
\begin{align}
    \ln(S_{t+\Delta}) &= \ln(S_t) + K_0 + K_1v_t + K_2v_{t+\Delta} + \sqrt{K_3v_t + K_4v_{t+\Delta}}\cdot Z \notag \\
    K_0 &= -\frac{\rho\kappa\theta}{\sigma}\Delta \notag \\
    K_1 &= \xi_1\Delta\left(\frac{\kappa\rho}{\sigma} - \frac{1}{2}\right)-\frac{\rho}{\sigma} \notag \\
    K_2 &= \xi_2\Delta\left(\frac{\kappa\rho}{\sigma}-\frac{1}{2}\right)+\frac{\rho}{\sigma} \notag \\
    K_3 &= \xi_1\Delta(1-\rho^2) \notag \\
    K_4 &= \xi_2\Delta(1-\rho^2) \notag
\end{align}
where $Z$ is standard normally distributed, and $\xi_1$ and $\xi_2$ are constants. The paper suggests $\xi_1 = \xi_2 = 0.5$.