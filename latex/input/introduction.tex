\chapter{Introduction}
\label{sec:introduction}

In many introductory finance courses, it is often assumed that returns in financial markets, particularly in the stock market, follow a normal distribution. However, this assumption is not supported by empirical evidence, as numerous studies (e.g., \cite{mandelbrotVariationCertainSpeculative1997,hullValueRiskWhen1998,karoglouBreakingNonnormalityStock2010}) have demonstrated that returns frequently deviate from normality. In particular, returns are often leptokurtic, meaning that their distribution exhibits fatter tails than the normal distribution. Additionally, returns tend to exhibit skewness, implying that their distribution is asymmetric. These properties are of great importance as they have significant implications for risk assessment and risk management.

The idea of normally distributed returns originates from Bachelier (\citeyear{bachelierTheorySpeculation1900}), who proposed that asset prices evolve according to a Brownian motion. This concept was later expanded by Black, Scholes, and Merton (\citeyear{blackPricingOptionsCorporate1973}) in their development of the Black-Scholes-Merton model, which revolutionized modern finance (\cite{heimerGenesisBlackScholesOption2008}). The British mathematician Ian Stewart even argues that the widespread reliance on this model played a role in the 2007–2008 financial crisis (\cite{stewartPursuitUnknown172012}).

One of the key weaknesses of the Black-Scholes model lies in its assumptions, particularly the assumptions of constant volatility and normally distributed returns. Empirical studies show that volatility is not constant and that returns often deviate from normality. To address these shortcomings, alternative models have been developed, including the Heston model, introduced by Heston (\citeyear{hestonClosedFormSolutionOptions1993}). The Heston model belongs to the class of stochastic volatility models, where volatility itself follows a stochastic process rather than remaining fixed. Due to its ability to better capture empirical characteristics of financial returns, the Heston model has become a widely used framework in financial mathematics.

Unlike the Black-Scholes model, the Heston model does not have a closed-form solution. Consequently, numerical methods must be employed to compute option prices. These methods are computationally intensive and typically involve either simulating a large number of price paths or using the characteristic function in combination with an inverse Fourier transform to obtain the risk-neutral density function (\cite{gatheralVolatilitySurfacePractitioner2011}). This density function can then be used to value derivative contracts, such as options.

A fundamental prerequisite for pricing options under the Heston model is the calibration of its parameters. The model can be fitted to market data by formulating it as a least-squares optimization problem, where the objective function aims to minimize the difference between model-implied prices and observed market prices. This calibration process is commonly performed using prices from vanilla options (\cite{flochAdaptiveFilonQuadrature2018}) or variance swaps (\cite{guillaumeHestonModelVariance2013}).

Since the return distribution under the Heston model is still relatively close to the normal distribution, this thesis explores the potential of expansion methods originally developed for the normal distribution, such as the Gram-Charlier expansion (\cite{gramUeberEntwickelungReeller1883,charlierContributionsMathematicalTheory1914}), to modify the normal distribution in a way that better captures the properties of Heston-simulated returns. To achieve this, the Heston model is simulated using Andersen's (\citeyear{andersenEfficientSimulationHeston2007}) Quadratic Exponential (QE) scheme over a large grid of parameter values. The realized moments and cumulants of the simulated data are computed, and various expansion techniques are applied. The resulting approximations are then compared to the theoretical density to assess how well these methods capture the true distribution. Moreover, this study aims to provide insights into improving pricing models and risk management strategies by identifying which expansion methods best reflect the statistical properties of asset returns.

This thesis is structured into several key sections, each building upon the previous to provide a comprehensive analysis of expansion methods for estimating probability density functions in the Heston stochastic volatility model. The study begins with a literature review, presented in the next chapter, which offers an overview of existing approaches for estimating realized moments and various expansion techniques found in the literature. While some methods are only briefly mentioned, the focus is on those that are directly used in this research.

The methodological framework consists of three key chapters. Chapter \ref{sec:moments} introduces realized moments, essential statistical measures for capturing the empirical properties of asset returns using high-frequency financial data. To assess their accuracy, Chapter \ref{sec:heston_model} presents the Heston stochastic volatility model, which accounts for stochastic volatility, a key feature of real-world asset prices. Finally, Chapter \ref{sec:expansion_methods} focuses on Gram-Charlier and Edgeworth expansions, which approximate probability density functions by incorporating higher-order cumulants, effectively capturing skewness and excess kurtosis.

Chapter \ref{sec:methodical_approach} integrates the concepts from the previous sections and describes the practical implementation of the study. The analysis and evaluation of the simulation results are presented in Chapter \ref{sec:results}. Finally, the thesis concludes with Chapter \ref{sec:conclusion_future_work}, which provides a summary of the key findings and discusses their implications for financial modeling. Additionally, the chapter outlines potential directions for future research.