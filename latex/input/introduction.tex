\section{Introduction}

- In vielen Einführungsvorlesungen in Finance wird die Annahme getroffen, dass Renditen am (Aktien-)Markt normalverteilt sind. Diese Annahme ist jedoch nicht korrekt, da empirische Untersuchungen zeigen (z.B. Mandelbrot 1997), dass Renditen oft nicht normalverteilt sind. So sind Renditen oft leptokurtisch, d.h. die Verteilung hat dickere Enden als die Normalverteilung. Zudem sind Renditen oft schief, d.h. die Verteilung ist nicht symmetrisch. Diese Eigenschaften sind wichtig, da sie die Risikobewertung und das Risikomanagement beeinflussen.
- Die Idee der normalverteilten Renditen kommt von Bachelier (1900), der die These aufstellte, dass der Preis eines Assests sich wie eine Brownian Motion verhält. Diese These wurde von Black, Scholes und Merton im Jahr 1973 weiterentwickelt zum Black-Scholes-Merton-Modell, welches die Finanzwelt revolutionierte (Heimer & Arend 2008). Der britische Mathematiker Ian Steward is sogar der Meinung, dass dieses Modell verantwortlich für die Finanzkrise 2007-2008 war (Steward 2012).
- Die Schwächen des Black-Scholes-Modells sind die Annahmen über unter anderem eine konstante Volatilität und eine normalverteilte Rendite. Diese Annahmen sind empirisch nicht haltbar, da die Volatilität oft nicht konstant ist und die Renditen oft nicht normalverteilt sind. Daher wurden Modelle entwickelt, die diese Schwächen adressieren. Eines dieser Modelle ist das Heston-Modell, welches von Heston (1993) vorgestellt wurde. Das Heston-Modell ist ein stochastisches Volatilitätsmodell, d.h. die Volatilität ist nicht konstant, sondern folgt auch einem Zufallsprozess. Das Heston-Modell ist ein beliebtes Modell in der Finanzmathematik, da es die Schwächen des Black-Scholes-Modells adressiert und die empirischen Eigenschaften von Renditen besser abbildet.
- Das Heston-Modell hat keine geschlossene Lösung mehr, im Gegensatz zum Black-Scholes-Modell, daher müssen numerische Verfahren zur Lösung des Modells verwendet werden. Diese Verfahren sind aufwendig, entweder simuliert man viele Pfade oder man nutzt die charakteristische Funktion in Kombination mit einer inversen Fourier-Transformation, um eine Preisverteilung zu berechnen. Diese Preisverteilung kann dann genutzt werden, um Optionen zu bewerten (Gatheral 2011).
- Das setzt voraus, dass man bereits die Parameter des Modells kennt. Man kann das Modell an den Markt anpassen, indem man es als Least-Squares-Problem definiert, wobei die Zielfunktion ist, die Preise am Markt zu reproduzieren. Normalerweise nimmt man die Preise von vanilla options (Floc'h 2018) oder von variance swaps (Guillaume & Schoutens 2013).
- Da Verteilung der Renditen doch recht ähnlich zur Normalverteilung ist, ist es das Ziel der Arbeit zu untersuchen, inwiefern sich Expansionsverfahren für die Normalverteilung, wie z.B. die Gram-Charlier Expansion (Gram 1883, Charlier 1914) genutzt werden kann die Normalverteilung zu verändern, um die Renditen des simulierten Heston-Modells abzubilden. Dazu wird das Heston-Modell mittels Andersens (2008) QE-Verfahren für einen großen Parameterraum simuliert, Momente und Kumulanten berechnet und dann die Expansionsverfahren angewendet. Die Ergebnisse werden dann mit der theoretischen Dichte verglichen, um zu sehen, wie gut die Expansionsverfahren die Dichte approximieren können.

- Die Arbeit gliedert sich in die folgenden Bereiche: Im Kapitel \ref{sec:heston_model} wird das Heston-Modell vorgestellt, das Kapitel \ref{sec:moments} werden Grundlagen zu Momenten und Kumulanten gelegt und auf die Berechnung dieser bei Hochfrequenz-Handelsdaten eingegangen. Im Kapitel \ref{sec:expansion_methods} werden die Expansionsverfahren vorgestellt und in Kapitel \ref{sec:methodical_approach} die praktische Umsetzung der Arbeit erläutert. Es folgt Kapitel \ref{sec:results} mit Ergebnissen und die Arbeit schließt mit Kapitel \ref{sec:conclusion} ab, wo die Arbeit zusammengefasst und ein Ausblick gegeben wird.