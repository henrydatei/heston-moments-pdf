\chapter{Literature Review}
\label{sec:literature}

- In diesem Kapitel geht es um einen kurzen Überblick über verschiedene Ansätze, die es in der Literatur gibt, die aber nicht in dieser Arbeit verwendet werden. Das liegt in der Regel daran, dass es mittlerweile bessere Verfahren gibt, oder dass die Ergebnisse nicht erfolgreich genug waren. Die Verfahren, die in dieser Arbeit verwendet werden, werden in den nachfolgenden Kapiteln ausführlich behandelt.

\section{Realized Moments}
For the pricing of financial derivatives, it is crucial to know the moments of returns, particularly those of monthly or quarterly returns (\cite{barroRareDisastersAsset2006}). However, estimating the moments of such low-frequency returns can be challenging due to the limited number of observations available (\cite{neubergerSkewnessStockMarket2021}). Today, financial markets operate continuously, making it possible to obtain daily or even minute-level returns without difficulty. For example, the German stock index DAX is calculated every second (\cite{boersefrankfurtFunktioniertBoerse}). There are several approaches to estimating the moments of monthly or quarterly returns based on the moments of daily returns. One such method is proposed by Amaya et al. (\citeyear{amayaDoesRealizedSkewness2015}). In this approach, the variance of daily returns is estimated using the sum of squared returns. This idea is not new and was first introduced by Andersen \& Bollerslev (\citeyear{andersenAnsweringSkepticsYes1998}). Building upon this approach, the daily realized skewness and kurtosis can be computed using cubed and quartic returns, respectively. Thia estimator is consistent, but it does not caputure skewness coming from the leverage effect (\cite{galloDynamicTailRisk2024}). Zhang et al. (\citeyear{zhangTaleTwoTime2005}) reports that this approach can be highly biased and the bias depends on the sampling frequency. Liu et al. (\citeyear{liuRealizedSkewnessHigh2014}) propose a new estimator on the basis of the Amaya et al. estimator which is robust to the microstructure noise at ultra-high frequency level. To transition from daily realized moments to weekly or monthly moments, a moving average approach is applied. Choe \& Lee (\citeyear{choeHighMomentVariations2014}) use variation processes to estimate low-frequency moments, specifically the quadratic variation of a semimartingale $X$. From these, the higher moments of $R$ follow as expectation of the quadratic covariation of $R$ and $R^2$ or the quadratic variation of $R^2$. The estimation of low-frequency variance follows the same approach as Andersen \& Bollerslev (\citeyear{andersenAnsweringSkepticsYes1998}) and Amaya et al. (\citeyear{amayaDoesRealizedSkewness2015}). 

\section{Expansion Methods}