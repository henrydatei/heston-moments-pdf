\chapter{Literature Review}
\label{sec:literature}

This chapter provides a brief overview of various approaches found in the literature that are not used in this work. The exclusion of these methods is generally due to the availability of more advanced techniques or because their results have not been sufficiently successful. The methods that are employed in this study are discussed in detail in the following chapters.

\section{Realized Moments}
For the pricing of financial derivatives, it is crucial to know the moments of returns, particularly those of monthly or quarterly returns (\cite{barroRareDisastersAsset2006}). But these moments are equally important in trading strategies, hedging, and risk management, as they provide critical insights into the distributional characteristics of asset returns (\cite{brooksOptimalHedgingHigher2012,jorionRiskManagementLessons1999,sheuEffectiveOptionsTrading2011}). However, estimating the moments of such low-frequency returns can be challenging due to the limited number of observations available (\cite{neubergerSkewnessStockMarket2021}). Today, financial markets operate continuously, making it possible to obtain daily or even minute-level returns without difficulty. For example, the German stock index DAX is calculated every second (\cite{boersefrankfurtFunktioniertBoerse}). There are several approaches to estimating the moments of monthly or quarterly returns based on the moments of daily returns. One such method is proposed by Amaya et al. (\citeyear{amayaDoesRealizedSkewness2015}). In this approach, the variance of daily returns is estimated using the sum of squared returns. This idea is not new and was first introduced by Andersen \& Bollerslev (\citeyear{andersenAnsweringSkepticsYes1998}). Building upon this approach, the daily realized skewness and kurtosis can be computed using cubed and quartic returns, respectively. Thia estimator is consistent, but it does not caputure skewness coming from the leverage effect (\cite{galloDynamicTailRisk2024}). Zhang et al. (\citeyear{zhangTaleTwoTime2005}) reports that this approach can be highly biased and the bias depends on the sampling frequency. Liu et al. (\citeyear{liuRealizedSkewnessHigh2014}) propose a new estimator on the basis of the Amaya et al. estimator which is robust to the microstructure noise at ultra-high frequency level. To transition from daily realized moments to weekly or monthly moments, a moving average approach is applied. Choe \& Lee (\citeyear{choeHighMomentVariations2014}) use variation processes to estimate low-frequency moments, specifically the quadratic variation of a semimartingale $X$. From these, the higher moments of $R$ follow as expectation of the quadratic covariation of $R$ and $R^2$ or the quadratic variation of $R^2$. The estimation of low-frequency variance follows the same approach as Andersen \& Bollerslev (\citeyear{andersenAnsweringSkepticsYes1998}) and Amaya et al. (\citeyear{amayaDoesRealizedSkewness2015}). 

In addition to these methods, Neuberger and Payne developed an approach for estimating realized moments. Since this method is utilized in this study, we will examine it in more detail in Chapter \ref{sec:moments}.

\section{Expansion Methods}

The Gram-Charlier expansion and the Edgeworth expansion are among the most well-known methods. They allow for the approximation of the density of a normal distribution while incorporating additional terms for skewness and kurtosis. Since these methods play a central role in this study, we will examine them in detail in Chapter \ref{sec:expansion_methods}.

In addition to these expansion methods, there are also the Cornish-Fisher expansion and the Saddlepoint approximation.

The Cornish-Fisher expansion, introduced by Cornish \& Fisher (\citeyear{cornishMomentsCumulantsSpecification1938}), is an asymptotic expansion that approximates the quantiles of a probability distribution based on its cumulants.

Given that $z_p$ is the $p$-quantile of a normal distribution with mean $\mu$ and variance $\sigma^2$, the $p$-quantile of a random variable $X$, denoted as $x_p$, can be approximated as follows (only the first terms shown, as it is common practice) (\cite{abramowitzHandbookMathematicalFunctions1968}, p. 935):
\begin{align}
    x_p \approx z_p + \frac{\gamma_1}{6}He_2(z_p) + \frac{\gamma_2^*}{24}He_3(z_p) - \frac{\gamma_1^2}{36}(2\cdot He_3(z_p) + He_1(z_p)) \notag
\end{align}
To obtain the probability density function (PDF), the quantiles $x_p$ can be numerically computed and differentiated. $He_n(x)$ are the Hermite polynomials, and $\gamma_1$ and $\gamma_2^*$ are the skewness and excess kurtosis of the approximated distribution, respectively.

Aboura \& Maillard (\citeyear{abouraOptionPricingSkewness2016}) point out that the parameters $\gamma_1$ and $\gamma_2^*$ do not correspond to the skewness and excess kurtosis of the approximated distribution. Instead, they denote these parameters as $s = \gamma_1$ and $k = \gamma_2^*$ and provide equations to compute the actual skewness $s^*$ and excess kurtosis $k^*$ of the approximated distribution. Later, Maillard (\citeyear{maillardUserGuideCornish2018}) published the inverse transformation, allowing one to compute the parameters $s$ and $k$ given the actual skewness and excess kurtosis. A corresponding table can be found in the appendix of his paper.

Aboura \& Maillard (\citeyear{abouraOptionPricingSkewness2016}) also investigate the domain of validity for the Cornish-Fisher expansion. Their findings suggest that the expansion is valid for a wide range of parameters—even an excess kurtosis above 40 and skewness exceeding $\pm 3$ are possible. However, when operating outside the validity domain, the issue is not immediately apparent in the probability density function itself. Instead, it becomes visible in the quantiles, which can turn negative.

The Saddlepoint Approximation, introduced by Daniels (\citeyear{danielsSaddlepointApproximationsStatistics1954}), provides an accurate method for approximating probability densities. While Daniels initially derived the density function, the cumulative distribution function (CDF) was later introduced by Lugannani \& Rice (\citeyear{lugannaniSaddlePointApproximation1980}). This method is based on the moment generating function (MGF) and offers a highly precise approximation formula. Given that $M(t)$ is the moment generating function and $K(t) = \log(M(t))$ is the cumulant generating function, the approximation of the density function $f(x)$ is given by:
\begin{align}
    \label{eq:sp_approximation}
    f(x)_{SP} \approx \frac{1}{\sqrt{2\pi\cdot K''(s)}}\exp(K(s) - s\cdot x)
\end{align}
where $s$ is the solution of the equation $K'(s) = x$.

By definition, the cumulant generating function $K(s)$ can be approximated with a Taylor series in $s$ which can then be used to calculate the Saddelpoint Approximation analytically.

The key advantages of this method are the high accuracy: The Saddlepoint method provides excellent approximations, even in distribution tail areas (\cite{duSystemReliabilityAnalysis2010,reidSaddlepointMethodsStatistical1988}). Furthermore there are no issues with negativity: The Saddlepoint Approximation does not produce negative densities. This is because the $\exp(\cdot)$ term in Equation \eqref{eq:sp_approximation} is always positive, and the denominator involves a square root, which is either positive or complex. If complex, the approximation does not exist, rather than producing invalid results.